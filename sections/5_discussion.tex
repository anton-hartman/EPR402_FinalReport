%%
%%  Department of Electrical, Electronic and Computer Engineering.
%%  EPR400/2 Final Report - Section 5.
%%  Copyright (C) 2011-2021 University of Pretoria.
%%

\section{Discussion}

\subsection{Critical evaluation of the design}

\subsubsection{Interpretation of results}
\paragraph{Laser turret}\hfill\\
It can be seen in \autoref{fig:q2_laser_to_setpoint} that laser turret responds rapidly to a step input. The laser turret reaches a settling point within 600\,ms of the step input for all runs in \autoref{fig:q2_laser_to_setpoint} other than run 3, which reached a settling point within 1200\,ms. The system specification requires the laser turret to reach a settling point within 2 seconds. The response time of the laser turret is well within this specification.

In \autoref{fig:q2_laser_errors}, it can be seen that the laser turret reaches the setpoint accurate to within 6 pixels for all tests with one exception. The system specification requires the laser turret to reach the setpoint accurate to within 1 millimetre. To determine whether this accuracy has been met the following measurements were made: the radius of the laser beam was measured to be a minimum of 5 pixels and the radius of a known 2-millimetre disc was measured to be an average 2 pixels. These two measurements indicate that the laser can be a maximum of 6 pixels away from the setpoint and still be within 1 millimetre of the setpoint. This indicates that all the test runs in \autoref{fig:q2_laser_errors}, except for one which has an error of 8 pixels, reach the setpoint accurate to within 1 millimetre. However, the radius of the laser beam and the know 2-millimetre disc do vary. The laser beam radius varies with the specific lighting conditions and the incident angle with the mosquito enclosure. The radii of the laser beam and disc also vary depending on the location in the camera frame in terms of the depth which they are from the camera and the lateral position in the camera frame since the camera is subject to lens distortion. This means that pixel distance that equates to 1 millimetre accuracy is not constant. The measurements of laser beam and disc radii were made conservatively such that the accuracy of the laser turret is likely to be better than 1 millimetre if it is within 6 pixels of the setpoint.

In run 1 of \autoref{fig:q2_laser_to_setpoint}, it can be seen that the laser plateaus for about 300\,ms before reaching the target position. This is likely due to a false negative laser detection. The laser position error is not updated for the period of false negative detections and thus no control action is taken since the control system is unaware of the error.


\paragraph{Mosquito detection}\hfill\\
From the bar graph in \autoref{fig:detection} it can be seen that the detection system produced 599 true positives, 0 false positives, 100 true negatives, and 1 false negative for the tests conducted. The system specification requires the detection system to have an accuracy of 90\% and a false positive rate of less than 5\%. The detection results indicate that the detection system has a 99\% accuracy and 0\% false positive rate. This is well within the system specification. However, the detection system was tested using a small set of specimens that were available. The specimens were dead mosquitoes and other similar flying insects that were glued to white shafts. The white shafts were moved by hand to mimic live mosquitoes. This test setup does not accurately represent the true appearance of live mosquitoes. The white shafts form shadows and other artefacts that are not present when detecting live mosquitoes. This reduces the range of pixel intensity values that can be considered for a potential mosquito detection. Therefore, the detection system could perform better with the given test specimens if it was not necessary to attach them to the white shafts.

\paragraph{Mosquito tracking}\hfill\\
In \autoref{fig:q4_tracking}, it can be seen that system is able to track multiple mosquitoes simultaneously. The tracking specification requires that mosquitoes are correctly associated between frames with 90\% accuracy. It can be seen by inspecting the systems tracked paths of the mosquitoes in \autoref{fig:q4_tracking} that the system has correctly associated the mosquitoes between frames with 100\% accuracy. This was determined by analysing the different flight paths captured. It can be seen that there is no other logical association of the flight paths other than the one shown in \autoref{fig:q4_tracking}. This system is able to correctly associate mosquitoes between frames with flight paths that cross each other. These results indicate that the tracking system meets the system specification. However, the tracking results were obtained using a small set of specimens that were moved by hand to mimic the true flight of mosquitoes. This does not accurately represent the true flight and behaviour of mosquitoes. The true flight of mosquitoes is likely to be more erratic than the mimicked flight. Real mosquitoes are also capable of crossing paths with trajectories that can lead to ambiguous identities that can result in incorrect association. However, for the performance of the overall system this should not be a concern since the ultimate goal of the system would be to kill all mosquitoes in the enclosure.

The tracking system does predict the future position of the mosquitoes, however it can be seen in \autoref{fig:q4_tracking} that when movement of the mosquitoes are highly non-linear this prediction is not accurate. This is because the system does not accurately model the movement of the mosquitoes. The acceleration of the mosquitoes are modelled are system noise since the acceleration of the mosquitoes cannot be accurately model due to their erratic movement.

\paragraph{Overall system performance}\hfill\\
The graphs in \Cref{fig:q1_1mos_detection_10fps,fig:q1_1mos_prediction_10fps,fig:q1_2mos_prediction_10fps} illustrate the performance of the system as a whole. \autoref{fig:q1_2mos_prediction_10fps} demonstrates the system's ability to track multiple mosquitoes simultaneously while targeting the mosquitoes with the laser. When multiple mosquitoes are tracked the system only targets the selected mosquito with the laser and ignores the other mosquitoes. The system can be seen to correctly target a single mosquito in the presence of other mosquitoes in \autoref{fig:q1_2mos_prediction_10fps}. The accuracy of the targeting system can be seen in \autoref{fig:q1_1mos_prediction_10fps}. In \autoref{fig:q1_1mos_prediction_10fps}, it can be seen that the laser is able to consistently remain near the position of the mosquito regardless of the movement of the mosquito. However, the laser is only able to illuminate the mosquito when it moves linearly for more or less 1 second. The primary requirement of the system is that the system must track mosquitoes in the mosquito enclosure and illuminate a mosquito every 5 seconds. The system is able to track mosquitoes in the mosquito enclosure, however, the system is only able to illuminate mosquitoes under certain conditions. The mosquito must move linearly or be stationary for a sufficient period of time, which is typically not more than 2 seconds. Therefore, if the mosquito does not exhibit this behaviour at least once every 5 seconds the system will fail to illuminate a mosquito according to the 5-second specification. Mosquitoes typically do not exhibit this behaviour during flight. They tend to move erratically or come to a complete rest. This means that for the typical behaviour of a mosquito the system will only be able to illuminate it when it has come to a rest.

\subsubsection{Critical evaluation}
The decision to use a segmentation approach rather than a deep learning approach for the mosquito and laser detection image processing proved to be a good decision. The image segmentation approach enabled the system to be optimised to perform most of the image processing operations on the Nvidia Jetson Nano \gls{gpu}. The image segmentation approach was able to achieve a 99\% accuracy in mosquito detection, while maintaining a low enough computational complexity to enable the system to operate in real-time near 10 frames per second.

The system was implemented to the capability to adjust many of the system's parameters at runtime. This enabled critical system parameters that determine the behaviour of the system to be tuned at runtime. The system also has multiple display options with different levels of granularity. The combination of the various display options and runtime adjustment of critical system parameters proved to be a powerful tool for debugging and optimising the system. This enabled individual parameters to be tuned while visualising the system at a level of granularity where its impact could be directly observed.

\subsubsection{Unsolved problems}\label{sec:unsolved_problems}
The system does not account for the slight difference in perspectives between the camera and the laser. This difference in perspective is inevitable due to the physical separation of the camera and the laser required to ensure that the laser and camera do not obstruct each other's point of view of the mosquito enclosure. The difference in perspective results two different physical locations of the laser beam corresponding to the same two-dimensional point in the camera's pixel co-ordinate frame. The two physical locations correspond to the point in the $xy$-plane in which the mosquito is and the point on the $xy$-plane of the back wall of the mosquito enclosure. The laser turret will always settle on the physical location that corresponds to the point on the back wall of the mosquito enclosure. This is not a problem when the difference in depth between the $xy$-plane of the back wall of the enclosure and the $xy$-plane in which the mosquito is is small. Through testing the maximum depth at which this difference is not problematic was found to be approximately half the depth of the mosquito enclosure. To resolve this problem it is necessary to determine the depth of the mosquito which is being tracked. This requires additional hardware such as a depth camera or a second camera with a known separation from the first camera. The depth of the mosquito can then be determined using triangulation.

The feedback from the camera based laser detection system is the only mechanism that the system has to sense the position of the laser turret's motors. This means that if the laser beam is outside the video frame the system has no way of knowing the position of the laser turret. This is problematic when the target position of the laser turret is near the edge of the video frame. When the target position is near the edge of the video frame the laser beam is likely to go outside the video frame due to overshoot and system inaccuracies. When this happens the system is unable to determine the required control action to return the laser to within the video frame since the feedback required to determine this control action is unavailable. This problem can be resolved by adding position encoders to the motors of the laser turret. This will enable the system to determine the position of the laser turret even when the laser beam is outside the video frame, albeit with less accuracy.


\subsubsection{Strong points of the design}
The system's internal \gls{led} lighting is very effective at illuminating the mosquito enclosure. This makes the system fairly robust to varying ambient lighting conditions. The system positioning bracket adds a level of robustness against accidental shifts in the relative positions of the system's hardware components.

The multi-threaded implementation of the system allowed the core system processes that perform the mosquito detection, laser detection, and mosquito tracking to operate in parallel with the two-axis \gls{pid} control required to operate the laser turret. The lightweight and optimised implementation of the system's core processing functions enabled the real-time operation of the system between 5 and 30 frames per second depending on state of the system and the amount of mosquitoes in the enclosure.

\subsubsection{Expected failure conditions}
The mosquito detection system relies on the difference in pixel intensity between the background of the mosquito enclosure and the mosquito. The system requires multiple pixels to be below a certain intensity threshold to detect a mosquito. This is to prevent false positives due to noise. This means that the system will fail to detect mosquitoes that are too minuscule since there will no or not enough pixels below the intensity threshold required to detect a mosquito. A similar failure condition exists for the laser detection system. If the laser beam is too small or not bright enough the system will fail to detect it. The effective laser beam size and brightness from the perspective of the camera varies depending on the angle with which it is incident on the mosquito enclosure. If the mosquito enclosure lighting is too bright it also causes the laser detection system to fail. This can be resolved by using a brighter laser at the expense of safety.

\subsection{Considerations in the design}
\subsubsection{Ergonomics}
The system has been designed to be modular to facilitate easy transportation of the system. The system is easy to assemble with the correct placement of components with the use of the system positioning bracket that can be seen in \autoref{fig:system_positioning_bracket}. The system has user-friendly \glspl{gui}. The main \gls{gui} displays the mosquito tracking and laser targeting in real-time as well as critical system information. The system enables manual control of the laser turret via the terminal. The system has a host of parameters that can be adjusted at runtime to optimise the performance of the system for the field conditions. The system also has numerous display options with different levels of granularity to facilitate debugging and optimisation of the system. At any stage of operation the user can type `\texttt{?}' into the terminal to view the current commands available.

\subsubsection{Health and safety}
Direct eye contact with the laser beam can be harmful to the user. A low-powered laser diode was used to minimise the risk of eye damage. A laser activation circuit was also designed to mitigate the risk of accidental laser eye contact. The laser is powered off by default and must be activated by the user through the terminal. The laser can easily be deactivated through the terminal with the single key command `\texttt{l}'. The presence of live mosquitoes in the mosquito enclosure poses a health risk to the user. The mosquito enclosure is designed with mesh openings that can easily be opened and closed to allow access the enclosure with minimal risk of mosquitoes escaping.

\subsubsection{Environmental impact}
The system does not create any pollutants or harmful emissions. If the system reaches its end of life, the system can be disassembled and most of its components can be recycled. The embedded platform, the motors, and the laser can be repurposed for other applications.

\subsubsection{Social and legal impact}
The system is developed as first concept. To actually kill mosquitoes the system will need to be equipped with a laser that can burn mosquitoes. This type of system would likely not be approved for public use. A laser this powerful presents a potential fire hazard and a potential safety hazard in the case of accidental eye contact or prolonged skin exposure. However, a successful implementation of such a system could prevent the spread of malaria and other mosquito-borne diseases, which would have a major positive impact on society. In summary there is much further research and development required before such a system could create an impact on society.

\subsubsection{Ethics clearance}
The system does not require ethics clearance.

\newpage

%% End of File.


